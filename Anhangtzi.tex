 \section{Anhang}
 \subsection{Detaillierte Zeitplanung}
 \label{app:Zeitplanung}

 \tabelleAnhang{tzi/Zeitplanungdetail.tex}


\clearpage
 \subsection{Use Case-Diagramm}\label{usecasediagram}

% Use Case-Diagramme und weitere \acs{UML}-Diagramme kann man auch direkt mit \LaTeX{} zeichnen, siehe \zB \url{http://metauml.sourceforge.net/old/usecase-diagram.html}.
 \begin{figure}[htb]
 \centering
 \includegraphicsKeepAspectRatio{usecaseb.pdf}{0.74}
 \caption{Use Case-Diagramm}
  \label{app:UseCase}

 \end{figure}

% \input{Anhang/AnhangPflichtenheft.tex}

% \subsection{Datenbankmodell}
% \label{app:Datenbankmodell}
% ER-Modelle kann man auch direkt mit \LaTeX{} zeichnen, siehe \zB \url{http://www.texample.net/tikz/examples/entity-relationship-diagram/}.
% \begin{figure}[htb]
% \centering
% \includegraphicsKeepAspectRatio{database.pdf}{1}
% \caption{Datenbankmodell}
% \end{figure}
 \clearpage
%\setcounter{section}{0}
%\subsection{Auszug aus dem Lastenheft}\label{lastenheft}

 \subsection{Lastenheft (Auszug)}
\label{app:Lastenheft}

Im folgenden Auszug aus dem Lastenheft werden die Anforderungen
definiert, die die Testumgebung erfüllen muss.

Die Anforderungen orientieren sich an einem minimalen Testszenario,
welches von Anforderer gewünscht ist: der Bestellung eines Produkts im
Onlineshop.

\textbf{Testszenario}:

\begin{itemize}
\itemsep1pt\parskip0pt\parsep0pt
\item
  Seite des Shop aufrufen
\item
  prüfen ob richtige Seite
\item
  Anmeldeformular finden
\item
  einloggen
\item
  Produktseite aufrufen
\item
  Produkt in den Warenkorb legen
\item
  zum Warenkorb navigieren
\item
  Adresse auswählen
\item
  Bezahlen
\item
  Bestellung abschicken.
\end{itemize}

\textbf{Fähigkeiten der Tests}

Daraus abgeleitet muss ein Test folgende Anforderungen erfüllen:

\begin{itemize}
\itemsep1pt\parskip0pt\parsep0pt
\item
  Seiten aufrufen
\item
  Seitenaufruf mit BasicAuth
\item
  HTTP und HTTPS Verbindungen
\item
  Überprüfen, ob DOM-Element auf Seite vorhanden.
\item
  Navigieren der Seite und Identifizierung von Elementen mit Hilfe von
  CSS-Selektoren\footnote{Syntax in Cascading Stylesheets, um Teile
    eines HTML-Dokumentes zu adressieren }, XPATH\footnote{XML
    Abfragesprache, um Teile eines XML-Dokumentes zu adressieren}, oder
  dem Inhalt von Elementen
\item
  Website anhand von Links navigieren
\item
  Javascript ausführen
\item
  Formulare wie z.B. das Anmeldeformular oder das Adressformular im Shop
  im Browser ausfüllen und absenden. Der Test soll die gleiche
  Javascript-basierte Validierung erfahren wie der Nutzer auch. Das
  bedeutete dass Formularinhalte client-seitig überprüft werden. Der
  einfache Versand von vorausgefüllten HTTP-POST-Request, ohne das
  Ausfüllen von Formularfeldern, genügt nicht.
\item
  Session-Handling
\item
  Logging des Testergebnis mit Ausgabe von Kommentaren im Testskript
\end{itemize}

\textbf{Fähigkeiten des Testrunner}

Zusätzlich zu den Fähigkeiten der Tests gibt es Anforderungen an den
Testrunner und seine Integration:

\begin{itemize}
\itemsep1pt\parskip0pt\parsep0pt
\item
  Der Testrunner muss auf mindestens einer der bestehenden Servern
  lauffähig sein (Syseleven: Gentoo Linux, md Rechenzentrum Düsseldorf:
  Suse Linux oder Ubuntu, Amazon EC2 virtuelle Instanz mit Ubuntu)
\item
  Die Testläufe müssen aus dem CD-System \emph{Go} ausgelöst werden
  können.
\item
  Die Testumgebung muss Erfolg oder Misserfolg eines Tests als
  Rückgabewert liefern können.
\item
  Die Testumgebung muss im Fehlerfall oder auf explizite Anweisung
  Screenshots der Seite erstellen
\item
  Es können Testbedingungen für den Test bereitgestellt werden. z.B. in
  Form von Datenbankabfragen die vorab Testnutzer und Testdaten anlegen
  oder wiederherstellen.
\item
  Eine Testsuite muss aus dem CD-System heraus gewählt werden können.
\item
  Es können Parameter aus dem CD-System an das Testsystem zur
  verfeinerten Steuerung der Testtiefe oder Auswahl von Tests weiter
  gereicht werden.
\item
  Es muss eine Historie von Testergebnissen und Artefakten und Testlogs
  von alten Testläufen aufbewahrt werden
\item
  Es müssen aktuelle Testskripte aus der Versionsverwaltung
  \emph{svn.gravis.de} ausgecheckt werden können
\item
  Es muss eine maschinenlesbare Auswertung von Testläufen erstellt
  werden.
\end{itemize}

\textbf{Abgrenzung}

\begin{itemize}
\itemsep1pt\parskip0pt\parsep0pt
\item
  Tests müssen während der Laufzeit nicht grafisch angezeigt werden
\end{itemize}

 \clearpage
% \subsection{Pflichtenheft}\label{pflichtenheft}
 %\section{Pflichtenheft (Auszug)}\label{pflichtenheft-auszug}

\begin{quote}
//ich bin ein Lösungskonzept, quasi, glaube ich
\end{quote}

Im folgenden Auszug aus dem Pflichtenheft wird die geplante Umsetzung
der im Lastenheft definierten Anforderungen beschrieben

\subsubsection{Umsetzung der Anforderungen
Test-Runtime}\label{umsetzung-der-anforderungen-test-runtime}

\begin{quote}
// Todo: habe wohl manchmal ghost und ghostjs geschrieben, ist falsch,
das ist was anderes. Alles so Gespensterworte, da kommt man
durcheinander.
\end{quote}

\begin{itemize}
\item
  Als Browser der Testumgebung wird PhantomJS\footnote{phantomjs.org/headless-testing.html}
  gewählt. Die Installation erfolgt über die Paketverwaltung der
  jeweiligen Distribution, das heißt portage\footnote{packages.gentoo.org/package/sys-apps/portage}
  für Gentoo, APT\footnote{wiki.ubuntuusers.de/APT} auf Ubuntu,
  brew\footnote{brew.sh} auf MacOS.
  \texttt{ich\ möchte\ das\ extra\ ansprechen,\ muss\ aber\ nicht\ im\ Pflichtenheft\ sein\ -\textgreater{}}
  Es werden die Binärdateien aus den jeweiligen Repositories genutzt.
  Zum Projektzeitpunkt wird PhantomJS 1.9.8 verteil. PhantomJS sollte
  nicht selbst kompiliert werden da es enorm viele Abhängigkeiten hat,
  was viele zusätzliche Fehlerquellen mit sich ziehen kann, und weil der
  Kompiliervorgang auch einem modernen Applikationsserver mehrere
  Stunden dauert. Da PhantomJS 2.0.0 auf dem Macintosh noch nicht
  startet und für linux nicht compiliert und die Tests und Testsuiten
  auf solchen Rechnern erstellt werden sollen, ist es empfohlen bei der
  stabilen Version 1.9.8 zu bleiben die sowohl unter Linux als auch Mac
  und Windows eingesetzt werden kann.

  \#asert you are on a 32 bit
  system\texttt{cd\ /usr/local/share\ sudo\ wget\ https://bitbucket.org/ariya/phantomjs/downloads/phantomjs-1.9.8-linux-i686.tar.bz2\ sudo\ tar\ xjf\ phantomjs-1.9.8-linux-i686.tar.bz2\ sudo\ ln\ -s\ /usr/local/share/phantomjs-1.9.8-linux-i686/bin/phantomjs\ /usr/local/share/phantomjs\ sudo\ ln\ -s\ /usr/local/share/phantomjs-1.9.8-linux-i686/bin/phantomjs\ /usr/local/bin/phantomjs\ sudo\ ln\ -s\ /usr/local/share/phantomjs-1.9.8-linux-i686/bin/phantomjs\ /usr/bin/phantomjs}
\item
  Als Test Runtime wird casperJS\footnote{} genutzt, es steht ebenfalls
  in den gängigen Repositories zur Verfügung und muss nicht selbst
  kompiliert werden. \textgreater{}//ätsch, in portage nicht, da muss
  man das selber auschecken. code steht hier temporär

  \$ git clone git://github.com/n1k0/casperjs.git \$ cd casperjs \$ ln
  -sf \texttt{pwd}/bin/casperjs /usr/local/bin/casperjs
\item
  casperJS wird mit einem \emph{tester} Modul ausgeliefert, welches für
  Unit- und funktionale Tests genutzt werden kann und eine API zur
  Verfügung stellt welche
  \texttt{vollständig?\ -\ muss\ ich\ das\ beweisen?} den Anforderungen
  an Testfähigkeiten aus dem Lastenheft genügt.
\item
  Tests werden in Javascript oder Coffeescript\footnote{coffescript.org
    coffescript ist eine Sprache die nach Javascript, genauer ECMAScript
    3, transcompiliert werden kann. Sie inspiriert sich von anderen
    Programmiersprachen wie Ruby oder Python und ist als syntactic sugar
    für JavaScript zu verstehen * Eine Ausgabe erfolgt von CasperJS}:
  geschrieben und nutzen die Funktionalität des \emph{tester} Modul von
  GhostJS zum abtasten der jeweiligen Seite und Funktionalität des
  \emph{utilis} Modul zum verarbeiten der Eingabeparameter und
  Ausgabe.\\Eine Aufnahme von Testabläufen ist mit den der
  Chrome-extension \emph{resurrectio}\footnote{https://github.com/ebrehault/resurrectio}
  möglich, diese gibt nach der Ausgabe den Quellcode eines Test als
  Javascript aus. \emph{resurrectio} wird seid 2013 nicht mehr gewartet
  sodass damit erstellte Tests nicht den vollen Funktionsumfang von
  Ghostjs \texttt{\$currentversion\ \ ,\ 1.0\ beta\ 3} ausnutzen können.
\item
  Test werden als JavaScript-Dateien gespeichert
\end{itemize}

Bildschrimfotos werden wie folgt erstellt.
https://github.com/casperjs/responsive-screenshots

\subsubsection{Umsetzung der Anforderungen Integration ins CI/CD
System}\label{umsetzung-der-anforderungen-integration-ins-cicd-system}

\begin{itemize}
\item
  Die Flusskontrolle \texttt{Die\ Logik\ ?} für Deployment- und
  Integrationsprozesse wird definiert mit Hilfe von ANT Skripten.
  Einzelne Aufgaben werden in sogenannte Pipelines verpackt.
  \texttt{Dieser\ Punkt\ kommt\ in\ den\ Flusstext,\ nicht\ Pflichtenheft}
\item
  Es wird in Go eine neue Pipelinegruppe angelegt mit je einer Pipeline
  pro zu testender Serverumgebung. Hier sind es 3 für das
  \emph{Integration}, \emph{Stage} und \emph{Echt}‌system.\\ Dazu werden
  neue Tags \texttt{XML-Knoten?} für Pipelines in die \texttt{go.xml}
  eingefügt. Der teaminternen Nomenklatur folgend heißen die Pipelines
  dann etwas ``UI.Test.ghostjs'' und ``UL.Test.ghostjs''
  \textgreater{}// Todo: Name klären
\item
  Es wird eine neue ghostjs.xml Datei erstellt die sämtliche ANT-Targets
  enthalten wird die für Regressionstests notwendig sind. Diese
  beinhalten u.a. :

  \begin{itemize}
  \itemsep1pt\parskip0pt\parsep0pt
  \item
    Auschecken aktueller Tests und Konfigurationsdateien aus der
    Versionsverwaltung ``svn.gravis.de/testing/trunk/ghostjs'' in das
    Basisverzeichnis der aktuellen Pipeline
  \item
    Vor-
  \item
    und nachbereitende Datenbankabfragen die Nutzerdaten von und für die
    Testnutzer der Regressionstests wie etwas Adressänderungen oder das
    löschen von Testkonten.
  \item
    Das Auslösen der Test
  \item
    Das Abholen und Aufbereitung der Testartefakte
  \item
    Die Auswertung der Testergebnisse
  \end{itemize}
\item
  Umgebungsvariablen werden in Go vergeben, die Basispfade für
  Testdateien und Artefakte festlegen, sowie Zugangsdaten für vor- und
  nachbereitende Datenbankzugriffe
\item
  Auf allen von Go verwalteten Servern muss ein ``Go-Agent'' ausgeführt
  werden. Dieser nimmt dann Arbeitsanweisungen von CI/CD System, aus den
  Pipelines entgegen und führt diese aus. Der Agent auf dem Server für
  Regressionstest muss in Go registriert werden.
\item
  Den Agenten muss eine oder mehrere Ressource(n) als Merkmal
  hinzugefügt werden

  \texttt{\textless{}agents\textgreater{}\ \ \ \ \ \textless{}agent\textgreater{}\ \ \ \ \ \ \ \ \ \textless{}name\textgreater{}mdsonline.stage.gravis.de.app2\textless{}/name\textgreater{}\ \ \ \ \ \ \ \ \ \textless{}ressources\textgreater{}stage\textless{}/ressources\textgreater{}\ \ \ \ \ \textless{}/agent\textgreater{}\ \textless{}agents\textgreater{}}
\item
  Zuweisung der ausführenden Ressource zu den jeweiligen Testing
  Pipelines.

  \texttt{\textless{}pipeline\textgreater{}\ \ \ \ \ \textless{}name\textgreater{}G.UI.testing.ghostjs\textless{}/name\textgreater{}\ \ \ \ \ \textless{}ressources\textgreater{}manager,stage\textless{}/ressources\textgreater{}\ \ \ \textless{}/pipeline\textgreater{}}
\item
  Das Auslösen der Test erfolgt über ein Unix-Shellkommando in einem
  ANT-Target. Hierbei wird casperJS mit Argumenten und Parametern
  aufgerufen. Hierbei können die Testtiefe als auch weitere Parameter
  übergebenwerden, etwa die gezielte Angebe von Suchtermini
  \texttt{\$ENV\_searchterms} oder Artikelnummern
  \texttt{\$ENV\_testporducts} die in die Test mit einbezogen werden
  sollen. Diese werden in Go als Umgebungsvariablen definiert und dann
  in das Unix-Shellkommando eingefügt.
  \texttt{casperjs\ Test\ pfad/zu/tests/*.js\ -\/-testdebth=\{\$ENV\_testdebth\}\ -\/-testproducts=\{\$ENV\_testporducts\}\ -\/-searchterms=\{\$ENV\_searchterms\}}.
  Damit ist bereits aus der Weboberfläche des CI/CD System eine hohe
  Anpassungsfähigkeit der Tests gewährleistet. Der Testleiter kann
  hierdurch ohne umständliche Code-Anpassungen die Testtiefe einstellen.
\item
  Es wird eine ``Pre-run'' Konfigurationsdatei für CasperJS die
  Zugangsdaten und wiederkehrende Prozeduren enthält sowie die
  Kommandozeilenparameter und -argumente aufbereitet und den folgenden
  Test in Variablen zur Verfügung stellt.
\item
  Ort für Artefakte bereiten
\item
  Ausgabe von Artefakten definieren (Screenshot)
\item
  Ausgabe von Artefakten definieren (XUnit-log)
\item
  Ausgabe von Artefakten definieren (Verbose Ausgabe von Casper
  \texttt{in\ Datei\ piepen?\ Kann\ ich\ ANSI-Farben\ behalten?})
\item
  \texttt{XUNIT-log\ schönen/aufbereiten?\ PHPUnit\ hat\ an\ dieser\ Stelle\ eine\ kleine\ Reportwebpage\ erstellt\ -\textgreater{}\ Will\ ich\ das,\ brauche\ ich\ das?}
\item
  Rückgabewerte an Go konkretisieren.
  \texttt{Schlägt\ die\ Pipeline\ fehl\ nur\ weil\ ein\ Test\ Fehlgeschlagen\ ist?}
\end{itemize}

% Hier Pflichtenheft
 %\clearpage

 \subsection{Bildschirmaufnahmen}\label{screenshots}
 Relevante Schritte der Bedienung des Testrunners in \emph{Go} sind in folgenden Bildschirmaufnahmen illustriert.

\begin{figure}[htb]
\centering
\includegraphicsKeepAspectRatio{Bilder/Edit_Pipeline_envvars.png}{0.83}
\caption{Konfigurieren von Umgebungsvariablen in der Administrationsoberfläche}
\label{fig:goguienvvars}
\end{figure}


 \begin{figure}[htb]
\centering
\includegraphicsKeepAspectRatio{Bilder/gouitrigger.png}{0.80}
\caption{Manuelles Auslösen einer Pipeline für Front-End-Tests in der Weboberfläche}
\label{fig:goguitrigger}
\end{figure}



\begin{figure}[htb]
\centering
\includegraphicsKeepAspectRatio{Bilder/Stage_Detail.png}{0.83}
\caption{  Hier wird zweite Stage der Pipeline UL.casperJStests ausgeführt.}
\label{fig:goguistagedetail}
\end{figure}

\begin{figure}[htb]
\centering

\includegraphicsKeepAspectRatio{Bilder/gather_artefacts_Job.png}{0.83}
\caption{Auswertung der Stage postprocessing }
\label{fig:goguisummary}
\end{figure}

\clearpage
  \subsection{Installationsprotokoll}\label{installprotocoll}
  %\newcommand{\hiddensubsection}[1]{
%    \stepcounter{subsection}
%    \subsection*{\arabic{chapter}.\arabic{section}.\arabic{subsection}\hspace{1em}{#1}}
%}
%Then you create your notoc subsection using this instead of \subsection:

%\hiddensubsection{sectionname}


\subsubsection*{PhantomJS}\label{phantomjs}

Als Browser der Testumgebung wird PhantomJS\footnote{phantomjs.org/headless-testing.html}
gewählt.\\Die Installation erfolgt über die Paketverwaltung der
jeweiligen Distribution, das heißt portage\footnote{packages.gentoo.org/package/sys-apps/portage}
für Gentoo, APT\footnote{wiki.ubuntuusers.de/APT} auf Ubuntu,
brew\footnote{brew.sh} auf MacOS X.\\Es werden die Binärdateien aus den
jeweiligen Repositories genutzt. Zum Projektzeitpunkt wird PhantomJS
1.9.8 verteilt.\\PhantomJS sollte nicht selbst kompiliert werden da es
enorm viele Abhängigkeiten hat, was viele zusätzliche Fehlerquellen mit
sich ziehen kann, und weil der Kompiliervorgang auch einem modernen
Applikationsserver mehrere Stunden dauert.\\Da PhantomJS 2.0.0 auf dem
Macintosh noch nicht startet und für linux nicht kompiliert, ist es
empfohlen bei der stabilen Version 1.9.8 zu bleiben die sowohl unter
Linux als auch Mac und Windows eingesetzt werden kann.

Als Server der alle zu testenden Webshops erreichen kann, wurde der
``manager'' Server als Installationsort ausgewählt. Auf dem
``manger''-Server, einem 32-bit System mit Gentoo Linux, war eine
Installation mit Hilfe von \texttt{portage} nicht möglich da es selber
kompilieren möchte, die Abhängigkeiten aber nicht alle aufgelöst werden
konnten. Der Hauptentwickler von PhantomJS bietet auf bitbucket
vorkompilierte PhantomJS Versionen an. Diese wird in das Verzeichnis
\texttt{/usr/local/share} extrahiert und anschließend werden Symlinks
dorthin in den Pfad gesetzt.

\begin{verbatim}
cd /usr/local/share
sudo wget https://bitbucket.org/ariya/phantomjs/downloads/phantomjs-1.9.8-linux-i686.tar.bz2
sudo tar xjf phantomjs-1.9.8-linux-i686.tar.bz2
sudo ln -s /usr/local/share/phantomjs-1.9.8-linux-i686/bin/phantomjs /usr/local/share/phantomjs
sudo ln -s /usr/local/share/phantomjs-1.9.8-linux-i686/bin/phantomjs /usr/local/bin/phantomjs
sudo ln -s /usr/local/share/phantomjs-1.9.8-linux-i686/bin/phantomjs /usr/bin/phantomjs
\end{verbatim}

\subsubsection*{casperJS}\label{casperjs}

Als Test-Runtime wird casperJS\footnote{casperjs.org} genutzt, es steht
ebenfalls in den Repositories der gängigen Paketverwaltungen zur
Verfügung und muss nicht selbst kompiliert werden. Unter MacOS X werden
die Quellen für \texttt{brew} aktualisiert, anschließend die neuste
developer-evaluation-Version installiert.

\begin{verbatim}
$ brew update
$ brew install casperjs --devel
\end{verbatim}

Unter gentoo Linux wird über die Paketverwaltung nur eine alte Version
von casperJS angeboten. Hier muss der aktuelle Code aus der
Versionsverwaltung geladen werden. Dazu wurde \texttt{git clone}
verwendet.

\begin{verbatim}
$ cd /usr/local
$ git clone git://github.com/n1k0/casperjs.git
$ cd casperjs
$ ln -sf `pwd`/bin/casperjs /usr/local/bin/casperjs
\end{verbatim}

\subsubsection*{Funktionstest}\label{funktionstest}

Es wurde überprüft ob die gewünschten Versionen der Software installiert
wurde.

\begin{verbatim}
it@manager ~ $ echo "phantomJS: $(phantomjs --version)" && echo "casperJS: $(casperjs --version)"
  phantomJS: 1.9.8
  casperJS: 1.1.0-beta3
\end{verbatim}

Die Funktionsfähigkeit von casperJS wird mit dem Selbsttest überprüft.

\begin{verbatim}
$ casperjs selftest
   […]
   PASS 1030 tests executed in 25.753s, 1030 passed, 0 failed, 0 dubious, 5 skipped.
   
\end{verbatim}

\subsubsection*{Konfiguration in \emph{Go}}\label{konfiguration-in-go}

Im der Weboberfläche von \emph{Go}, der Agentenkonfigurationsseite,
wurde dem Entsprechenden Agentenserver ``manager'' eine neue Ressource
mit dem Namen ``casperjs'' hinzugefügt. Erst jetzt kann \emph{casperJS}
auch aus Go heraus tatsächlich genutzt werden.

  \clearpage

\subsection{Liste der eingesetzen
Software}\label{liste-der-eingesetzen-software}

\begin{itemize}
\itemsep1pt\parskip0pt\parsep0pt
\item
  MacOS X
\item
  Gentoo GNU/Linux
\item
  Ubuntu GNU/Linux 14.04
\item
  Oracle VirtualBox
\item
  SublimeText2 als Texteditor und \acs{IDE}
\item
  git \& svn zur Versionsverwaltung der Projektsoftware, -Dokumentation
  und Tests.
\item
  phantomJS als headless Browser
\item
  casperJS als Testrunner
\item
  ghostdriver als Testrunner
\item
  Thoughtworks Go also CI/CD Platform
\item
  SeleniumIDE Plugin für Firefox
\item
  pandoc, pdflatex, MacTeX, MacDown, Mou zur Erstellung der
  Projektdokumentation
\end{itemize}

% \input{Anhang/AnhangEntwuerfe.tex}
% \clearpage
% \input{Anhang/AnhangScreenshots.tex}
% \input{Anhang/AnhangDoc.tex}
% \clearpage
% \input{Anhang/AnhangTest.tex}

% \subsection{Klasse: ComparedNaturalModuleInformation}
% \label{app:CNMI}
% Kommentare und simple Getter/Setter werden nicht angezeigt.
% \lstinputlisting[language=php]{Listings/cnmi.php}
% \clearpage

% \subsection{Klassendiagramm}
% \label{app:Klassendiagramm}
% Klassendiagramme und weitere \acs{UML}-Diagramme kann man auch direkt mit \LaTeX{} zeichnen, siehe \zB \url{http://metauml.sourceforge.net/old/class-diagram.html}.
% \begin{figure}[htb]
% \centering
% \includegraphicsKeepAspectRatio{Klassendiagramm.pdf}{1}
% \caption{Klassendiagramm}
% \end{figure}
% \clearpage

\subsection{Quellcode der ANT Targets}
\label{app:anttargets}
\lstinputlisting[language=xml]{Listings/casperjs.xml}

\subsection{Quellcode der Pipeline}
Hinweis: Aufgrund des eingeschränkten Umfangs dieser Dokumentation wurde an einigen Stellen die XML-Datei verkürzt dargestellt. Die entsprechenden Stellen wurden mit einem Auslassungszeichen ([...])gekennzeichnet. Firmeninterna wurden mit "XXX" überschrieben.
\label{app:casperpipeline}
\lstinputlisting[language=xml]{Listings/casperpipe.xml}


% \input{Anhang/AnhangBenutzerDoku.tex}
