% Seitenränder -----------------------------------------------------------------
\setlength{\topskip}{\ht\strutbox} % behebt Warnung von geometry
\geometry{a4paper,left=20mm,right=20mm,top=25mm,bottom=35mm}

\usepackage[
	automark, % Kapitelangaben in Kopfzeile automatisch erstellen
	headsepline, % Trennlinie unter Kopfzeile
	ilines % Trennlinie linksbündig ausrichten
]{scrpage2}

% Kopf- und Fußzeilen ----------------------------------------------------------
\pagestyle{scrheadings}
\newcommand{\chapterpagestyle}{scrheadings}
\clearscrheadfoot

% Kopfzeile
\renewcommand{\headfont}{\normalfont} % Schriftform der Kopfzeile
\ihead{\large{\textsc{\titel}}\\ \textit{\headmark}}
%\ihead{\large{\textsc{\titel}}\\ \small{\untertitel} \\[2ex] \textit{\headmark}}
\chead{}
\ohead{\includegraphics[scale=0.15]{\logo}}
\setlength{\headheight}{15mm} % Höhe der Kopfzeile
%\setheadwidth[0pt]{textwithmarginpar} % Kopfzeile über den Text hinaus verbreitern (falls Logo den Text überdeckt) 

% Fußzeile
\ifoot{\copyright\ \autor}
\cfoot{}
\ofoot{\pagemark}

\onehalfspacing % Zeilenabstand 1,5 Zeilen
\frenchspacing % erzeugt ein wenig mehr Platz hinter einem Punkt 

% Schusterjungen und Hurenkinder vermeiden
\clubpenalty = 10000
\widowpenalty = 10000 
\displaywidowpenalty = 10000

% Quellcode-Ausgabe formatieren
\lstset{numbers=left, numberstyle=\tiny, numbersep=5pt, breaklines=true}
\lstset{emph={square}, emphstyle=\color{red}, emph={[2]root,base}, emphstyle={[2]\color{blue}}}

\counterwithout{footnote}{section} % Fußnoten fortlaufend durchnummerieren
\setcounter{tocdepth}{\subsubsectionlevel} % im Inhaltsverzeichnis werden die Kapitel bis zum Level der subsubsection übernommen
\setcounter{secnumdepth}{\subsubsectionlevel} % Kapitel bis zum Level der subsubsection werden nummeriert

% Aufzählungen anpassen
\renewcommand{\labelenumi}{\arabic{enumi}.}
\renewcommand{\labelenumii}{\arabic{enumi}.\arabic{enumii}.}
\renewcommand{\labelenumiii}{\arabic{enumi}.\arabic{enumii}.\arabic{enumiii}}

% Tabellenfärbung:
%\definecolor{heading}{rgb}{0.475,0.647,0.082}
\definecolor{heading}{rgb}{0.647,0.729,0.37}
\definecolor{odd}{rgb}{0.9,0.9,0.9}
\definecolor{gragreen}{rgb}{0.737,0.89,0.23}
%\definecolor{gragreen}{rgb}{0.52,0.667,0}
\definecolor{dropdown}{rgb}{0.431,0.49,0.6}
