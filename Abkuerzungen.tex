% Es werden nur die Abkürzungen aufgelistet, die mit \ac definiert und auch benutzt wurden. 
%
% \acro{VERSIS}{Versicherungsinformationssystem\acroextra{ (Bestandsführungssystem)}}
% Ergibt in der Liste: VERSIS Versicherungsinformationssystem (Bestandsführungssystem)
% Im Text aber: \ac{VERSIS} -> Versicherungsinformationssystem (VERSIS)

% Hinweis: allgemein bekannte Abkürzungen wie z.B. bzw. u.a. müssen nicht ins Abkürzungsverzeichnis aufgenommen werden
% Hinweis: allgemein bekannte IT-Begriffe wie Datenbank oder Programmiersprache müssen nicht erläutert werden,
%          aber ggfs. Fachbegriffe aus der Domäne des Prüflings (z.B. Versicherung)

\begin{acronym}[TDMA]
	\acro{ANT}{Java Programm zum automatisierten Erzeugen und Deployen von ausführbaren Computerprogrammen aus Quelltexten}
	\acro{Ausspiel}{siehe Deployment}
	\acro{Deployment}{Ausieferung, ausrollen von Software}
	\acro{API}{Application Programming Interface}
	\acro{CI/CD}{Continuous Integration, Continuous Deployment}
	\acro{CI}{Continuous Integration}
	\acro{CD}{Continuous Deployment}
	\acro{CSS}{Cascading Style Sheets}
	\acro{DSL}{domain sepcific language}
	\acro{CSV}{Comma Separated Value}
	\acro{Echt-System}{Serverumgebung für den produktiven Einsatz}
	\acro{Staging-System}{produktionsnahe [Server-] Testumgebung für Abnahmen und Tests}
	\acro{Go}{Thoughtworks Go, Continuous Integration System}
	\acro{GRAVIS}{GRAVIS Computervertriebsgesellschaft mbH}
	\acro{headless}{Programmablauf ohne Ausgabe auf einen Bildschirm}
	\acro{HTML}{Hypertext Markup Language}\acused{HTML}
	\acro{HTTP}{Hypertext Transport Protocol}\acused{HTTP}
	\acro{IDE}{Integrated Development Environment}
	\acro{md}{mobilcom-debitel}
	\acro{PHP}{PHP Hypertext Preprocessor}
	\acro{RAM}{Random Access Memory}
	\acro{SDK}{Software Development Kit}
	\acro{SQL}{Structured Query Language}
	\acro{SVN}{Subversion}
	\acro{Pipeline}{Anweisungskette zur Abarbeitung sequentieller und konditionaler Arbeitsschritte}
	\acro{post-commit-hook}{Ereignis welches nach dem Commit in die Versionsverwaltung ausgelöst wird.}
	\acro{Stage}{Synchron ausführbarer Teil einer Pipeline}
	\acro{Target}{Sprungziel einer Pipeline in einer \acs{ANT} build.xml}
	\acro{Task}{elementare,ausführbare Aufgabe. Entspricht einer Programmmethode in ANT}
	\acro{Testsuite}{Zusamenhängende Sammlung von Tests}
	\acro{XML}{Extensible Markup Language}
\end{acronym}
