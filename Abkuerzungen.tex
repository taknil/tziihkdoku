% Es werden nur die Abkürzungen aufgelistet, die mit \ac definiert und auch benutzt wurden. 
%
% \acro{VERSIS}{Versicherungsinformationssystem\acroextra{ (Bestandsführungssystem)}}
% Ergibt in der Liste: VERSIS Versicherungsinformationssystem (Bestandsführungssystem)
% Im Text aber: \ac{VERSIS} -> Versicherungsinformationssystem (VERSIS)

% Hinweis: allgemein bekannte Abkürzungen wie z.B. bzw. u.a. müssen nicht ins Abkürzungsverzeichnis aufgenommen werden
% Hinweis: allgemein bekannte IT-Begriffe wie Datenbank oder Programmiersprache müssen nicht erläutert werden,
%          aber ggfs. Fachbegriffe aus der Domäne des Prüflings (z.B. Versicherung)

\begin{acronym}
	\acro{API}{Application Programming Interface}
	\acro{CI/CD}{Continuous Integration, Continuous Deployment}
	\acro{CI}{Continuous Integration}
	\acro{CD}{Continuous Deployment}
	\acro{CSS}{Cascading Style Sheets}
	\acro{CSV}{Comma Separated Value}
	\acro{Go}{Thoughtworks Go, Continuous Integration System}
	\acro{GRAVIS}{GRAVIS Computervertriebsgesellschaft mbH}
	\acro{HTML}{Hypertext Markup Language}\acused{HTML}
	\acro{IDE}{Integrated Development Environment}
	\acro{md}{mobilcom-debitel}
	\acro{Natural}[\textsc{Natural}]{Programmiersprache der Software AG}\acused{Natural}
	\acro{PHP}{PHP Hypertext Preprocessor}
	\acro{RAM}{Random Access Memory}
	\acro{SDK}{Software Development Kit}
	\acro{SQL}{Structured Query Language}
	\acro{SVN}{Subversion}
\end{acronym}
