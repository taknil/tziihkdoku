\subsection{Lastenheft (Auszug)}
\label{app:Lastenheft}

Es folgt ein Auszug aus dem Lastenheft mit Fokus auf die Anforderungen:

Die Anforderungen orientieren sich an einem minimalen Testszenario, der
Bestellung eines Produkts im Webshop.

\textbf{Testszenario}:

\begin{itemize}
\itemsep1pt\parskip0pt\parsep0pt
\item
  Seite des Shop aufrufen
\item
  prüfen ob richtige Seite
\item
  Anmeldeformular finden
\item
  einloggen
\item
  Produktseite aufrufen
\item
  Produkt in den Warenkorb legen
\item
  zum Warenkorb navigieren
\item
  Adresse auswählen
\item
  Bezahlen
\item
  Bestellung abschicken.
\end{itemize}

\textbf{Testfähigkeiten}

Daraus abgeleitet muss ein Test folgende Anforderungen erfüllen:

\begin{itemize}
\itemsep1pt\parskip0pt\parsep0pt
\item
  Seiten aufrufen
\item
  Seitenaufruf mit BasicAuth
\item
  HTTP und HTTPS Verbindungen
\item
  Überprüfen ob Element auf Seite vorhanden.
\item
  Navigieren der Seite und Identifizierung von Elementen mit Hilfe von
  XPATH\footnote{XML Abfragesprache, um Teile eines XML-Dokumentes zu
    adressieren}, CSS-Selektoren \footnote{Syntax in Cascading
    Stylesheets, um Teile eines HTML-Dokumentes zu adressieren } oder
  dem Inhalt von Elementen
\item
  Website anhand von Links navigieren
\item
  Javascript ausführen
\item
  Formulare wie z.B. das Anmeldeformular oder das Adressformular im Shop
  im Browser ausfüllen und absenden. Der Test soll die gleiche
  Javascript-basierte Validierung erfahren wie der Nutzer auch. Der
  einfache Versand von vorausgefüllten HTTP-POST-Request genügt nicht. 
\item
  Logging des Testergebnis mit Ausgabe von Kommentaren im Testskript
\end{itemize}

\textbf{Umgebungsfähigkeiten}

Zusätzlich zu den Fähigkeiten der Tests gibt es Anforderungen an die
Testumgebung und ihre Integration. :

\begin{itemize}
\itemsep1pt\parskip0pt\parsep0pt
\item
  Die Testmgebung muss auf mindestens einer der bestehenden
  Serverumgebungen lauffähig sein (Syseleven: Gentoo Linux, md
  Rechenzentrum Düsseldorf: Suse Linux oder Ubuntu, Amazon EC2 virtuelle
  Instanz mit Ubuntu)
\item
  Die Testläufe müssen aus dem CI/CD System \emph{Go} ausgelöst werden
  können.
\item
  Die Testumgebung muss Erfolg oder Misserfolg eines Tests als
  Rückgabewert liefern können.
\item
  Die Testumgebung muss im Fehlerfall oder auf explizite Anweisung
  Screenshots der Seite erstellen
\item
  Es können Testbedingungen für den Test bereitgestellt werden. z.B. in
  Form von Datenbankabfragen die vorab Testnutzer und Testdaten anlegen
  oder wiederherstellen.
\item
  Es können Parameter aus dem CI/CD-System an das Testsystem zur
  verfeinerten Steuerung der Testtiefe oder Auswahl von Testobjekten
  weiter gereicht werden.
\item
  Es muss eine Historie von Testergebnissen und Artefakten und Testlogs
  von alten Testläufen aufbewahrt werden
\item
  Es müssen aktuelle Testskripte aus der Versionsverwaltung
  \emph{svn.gravis.de} ausgecheckt werden können
\item
  Eine Testsuite muss aus dem CI/CD System heraus gewählt werden können.
\item
  Es muss eine maschinenlesbare Auswertung von Tests erstellt werden.
\end{itemize}
