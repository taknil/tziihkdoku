\subsection{Anforderungen}\label{anforderungen}

Im folgenden Auszug aus dem Lastenheft werden die Anforderungen
definiert, die die Testumgebung erfülen muss.

Die Anforderungen orientieren sich an einem minimalen/basis-
Testscenario, der Bestellung eine Produkt im Webshop, welches vom
Anforderer gewüscht ist.

Testscenario: Seite des Shop aufrufen, prüfen ob richtige Seite,
Anmeldeformular finden, einlogen, Produktseite aufrufen, Produkt in den
Warenkorb legen, zum Warenkorb navigieren, Adresse auswählen, Bezahlen,
Bestellung abschicken.

Daraus abgeleitete Funktionalitäten:

\subsubsection{Testfähigkeiten}\label{testfuxe4higkeiten}

\begin{itemize}
\itemsep1pt\parskip0pt\parsep0pt
\item
  Seiten aufraufen
\item
  Überprüfen ob Element auf Seite vorhanden.
\item
  Navigieren der Seite und Identifizierung von Elementen mit Hilfe von
  XPATH, CSS-Selektoren oder dem Inhalt von Elementen
\item
  Website anhand von Links navigieren
\item
  Javascript ausführen
\item
  Formulare wie z.B. das Anmeldeformular oder das Adressformular im Shop
  im Browser ausfüllen und absenden. Der Test soll die gleiche
  Javascript Validierung erfahren wie der Nutzer auch. Der einfache
  Versand von vorausgefüllten HTTP Post-Request genügt nicht.
\item
  HTTPS Verbindungen
\item
  Session-Handling für log-in
\item
  Logging des Testergebnis mit Ausgabe von Kommentagen im Test
\end{itemize}

\paragraph{Optional:}\label{optional}

\begin{itemize}
\itemsep1pt\parskip0pt\parsep0pt
\item
  Dateidownload
\item
  Dateiupload
\end{itemize}

\subsubsection{Umgebungsfähigkeiten}\label{umgebungsfuxe4higkeiten}

Zusätzlich zu den Fähigkeiten der Tests gibt es Anforderungen an die
Testumgebung und ihre Integration. Die Umgebung:

\begin{itemize}
\itemsep1pt\parskip0pt\parsep0pt
\item
  Muss auf mindestens einer der bestehenden Serverumgebungen lauffähig
  sein (Syseleven: Gentoo Linux, md Rechenzentrum Düsseldorf: Suse Linux
  oder Ubuntu, Amazon EC2 virtuelle Instanz mit Ubuntu)
\item
  muss aus dem CI/CD System angesprochen / gestartet / ausgelöst werden
  können.
\item
  muss Erfolg oder Fehlschlagen eines Test als Rückgabewert liefern
  können.
\item
  muss im Fehlerfall oder auf explizite Anweisung Screenshots der Seite
  erstellen, abspeichern und als Artefakt vorhalten können.
\item
  muss die Seite renden
\item
  kann Daten für den Test bereitstellen in Form von Datenbankabfragen
  die vorab Testnutzer und Daten anlegen oder wiederherstellen.
\item
  kann Parametern aus dem CI/CD-System an das Testsystem zur
  verfeinerten Steuerung der Testtiefe oder Auswahl von Testobjekten
  weiterreichen.
\item
  muss nicht den Test graphisch während der Laufzeit anzeigen
\item
  muss aus sich heraus oder in der Historie des CI/CD System
  Testergebnisse und Artefakte von alten Testläufen aufbewahren
\item
  muss aktuelle Tests, also deren Quellcode, aus der Versionsverwaltung
  \emph{svn.gravis.de} auschecken können
\end{itemize}
