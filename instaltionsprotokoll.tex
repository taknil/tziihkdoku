\subsubsection{PhantomJS}\label{phantomjs}

Als Browser der Testumgebung wird PhantomJS\footnote{phantomjs.org/headless-testing.html}
gewählt.\\Die Installation erfolgt über die Paketverwaltung der
jeweiligen Distribution, das heißt portage\footnote{packages.gentoo.org/package/sys-apps/portage}
für Gentoo, APT\footnote{wiki.ubuntuusers.de/APT} auf Ubuntu,
brew\footnote{brew.sh} auf MacOS X.\\Es werden die Binärdateien aus den
jeweiligen Repositories genutzt. Zum Projektzeitpunkt wird PhantomJS
1.9.8 verteilt.\\PhantomJS sollte nicht selbst kompiliert werden, da es
enorm viele Abhängigkeiten hat, was viele zusätzliche Fehlerquellen mit
sich ziehen kann, und weil der Kompiliervorgang auch einem modernen
Applikationsserver mehrere Stunden dauert.\\Da PhantomJS 2.0.0 auf dem
Macintosh noch nicht startet und für linux nicht kompiliert, ist es
empfohlen, bei der stabilen Version 1.9.8 zu bleiben die sowohl unter
Linux als auch Mac und Windows eingesetzt werden kann.

Als Server, der alle zu testenden Onlineshop erreichen kann, wurde der
``manager'' Server als Installationsort ausgewählt. Auf dem
``manager''-Server, einem 32-bit System mit Gentoo Linux, war eine
Installation mit Hilfe von \texttt{portage} nicht möglich, da es selber
kompilieren möchte, die Abhängigkeiten aber nicht alle aufgelöst werden
konnten. Der Hauptentwickler von PhantomJS bietet auf bitbucket
vorkompilierte PhantomJS Versionen an. Diese wird in das Verzeichnis
\texttt{/usr/local/share} extrahiert und anschließend werden Symlinks
dorthin in den Pfad gesetzt.

\begin{verbatim}
cd /usr/local/share
sudo wget https://bitbucket.org/ariya/phantomjs/downloads/phantomjs-1.9.8-linux-i686.tar.bz2
sudo tar xjf phantomjs-1.9.8-linux-i686.tar.bz2
sudo ln -s /usr/local/share/phantomjs-1.9.8-linux-i686/bin/phantomjs /usr/local/share/phantomjs
sudo ln -s /usr/local/share/phantomjs-1.9.8-linux-i686/bin/phantomjs /usr/local/bin/phantomjs
sudo ln -s /usr/local/share/phantomjs-1.9.8-linux-i686/bin/phantomjs /usr/bin/phantomjs
\end{verbatim}

\subsubsection{casperJS}\label{casperjs}

Als Test-Runtime wird casperJS\footnote{casperjs.org} genutzt, es steht
ebenfalls in den Repositories der gängigen Paketverwaltungen zur
Verfügung und muss nicht selbst kompiliert werden. Unter MacOS X werden
die Quellen für \texttt{brew} aktualisiert, anschließend die neuste
developer-evaluation-Version installiert.

\begin{verbatim}
$ brew update
$ brew install casperjs --devel
\end{verbatim}

Unter gentoo Linux wird über die Paketverwaltung nur eine alte Version
von casperJS angeboten. Hier muss der aktuelle Code aus der
Versionsverwaltung geladen werden. Dazu wurde \texttt{git clone}
verwendet. Anschließend ein symbolischer Link in ein Verzeichnis aus dem
Pfad erstellt.

\begin{verbatim}
$ cd /usr/local
$ git clone git://github.com/n1k0/casperjs.git
$ cd casperjs
$ ln -sf `pwd`/bin/casperjs /usr/local/bin/casperjs
\end{verbatim}

\subsubsection{Funktionstest}\label{funktionstest}

Es wurde überprüft, ob die gewünschten Versionen der Software
installiert wurde.

\begin{verbatim}
it@manager ~ $ echo "phantomJS: $(phantomjs --version)" && echo "casperJS: $(casperjs --version)"
  phantomJS: 1.9.8
  casperJS: 1.1.0-beta3
\end{verbatim}

Die Funktionsfähigkeit von casperJS wurde mit dem Selbsttest überprüft.

\begin{verbatim}
$ casperjs selftest
   […]
   PASS 1030 tests executed in 25.753s, 1030 passed, 0 failed, 0 dubious, 5 skipped.
   
\end{verbatim}
