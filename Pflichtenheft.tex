\section{Pflichtenheft (Auszug)}\label{pflichtenheft-auszug}

\begin{quote}
//ich bin ein Lösungskonzept, quasi, glaube ich
\end{quote}

Im folgenden Auszug aus dem Pflichtenheft wird die geplante Umsetzung
der im Lastenheft definierten Anforderungen beschrieben

\subsubsection{Umsetzung der Anforderungen
Test-Runtime}\label{umsetzung-der-anforderungen-test-runtime}

\begin{quote}
// Todo: habe wohl manchmal ghost und ghostjs geschrieben, ist falsch,
das ist was anderes. Alles so Gespensterworte, da kommt man
durcheinander.
\end{quote}

\begin{itemize}
\item
  Als Browser der Testumgebung wird PhantomJS\footnote{phantomjs.org/headless-testing.html}
  gewählt. Die Installation erfolgt über die Paketverwaltung der
  jeweiligen Distribution, das heißt portage\footnote{packages.gentoo.org/package/sys-apps/portage}
  für Gentoo, APT\footnote{wiki.ubuntuusers.de/APT} auf Ubuntu,
  brew\footnote{brew.sh} auf MacOS.
  \texttt{ich\ möchte\ das\ extra\ ansprechen,\ muss\ aber\ nicht\ im\ Pflichtenheft\ sein\ -\textgreater{}}
  Es werden die Binärdateien aus den jeweiligen Repositories genutzt.
  Zum Projektzeitpunkt wird PhantomJS 1.9.8 verteil. PhantomJS sollte
  nicht selbst kompiliert werden da es enorm viele Abhängigkeiten hat,
  was viele zusätzliche Fehlerquellen mit sich ziehen kann, und weil der
  Kompiliervorgang auch einem modernen Applikationsserver mehrere
  Stunden dauert. Da PhantomJS 2.0.0 auf dem Macintosh noch nicht
  startet und für linux nicht compiliert und die Tests und Testsuiten
  auf solchen Rechnern erstellt werden sollen, ist es empfohlen bei der
  stabilen Version 1.9.8 zu bleiben die sowohl unter Linux als auch Mac
  und Windows eingesetzt werden kann.

  \#asert you are on a 32 bit
  system\texttt{cd\ /usr/local/share\ sudo\ wget\ https://bitbucket.org/ariya/phantomjs/downloads/phantomjs-1.9.8-linux-i686.tar.bz2\ sudo\ tar\ xjf\ phantomjs-1.9.8-linux-i686.tar.bz2\ sudo\ ln\ -s\ /usr/local/share/phantomjs-1.9.8-linux-i686/bin/phantomjs\ /usr/local/share/phantomjs\ sudo\ ln\ -s\ /usr/local/share/phantomjs-1.9.8-linux-i686/bin/phantomjs\ /usr/local/bin/phantomjs\ sudo\ ln\ -s\ /usr/local/share/phantomjs-1.9.8-linux-i686/bin/phantomjs\ /usr/bin/phantomjs}
\item
  Als Test Runtime wird casperJS\footnote{} genutzt, es steht ebenfalls
  in den gängigen Repositories zur Verfügung und muss nicht selbst
  kompiliert werden. \textgreater{}//ätsch, in portage nicht, da muss
  man das selber auschecken. code steht hier temporär

  \$ git clone git://github.com/n1k0/casperjs.git \$ cd casperjs \$ ln
  -sf \texttt{pwd}/bin/casperjs /usr/local/bin/casperjs
\item
  casperJS wird mit einem \emph{tester} Modul ausgeliefert, welches für
  Unit- und funktionale Tests genutzt werden kann und eine API zur
  Verfügung stellt welche
  \texttt{vollständig?\ -\ muss\ ich\ das\ beweisen?} den Anforderungen
  an Testfähigkeiten aus dem Lastenheft genügt.
\item
  Tests werden in Javascript oder Coffeescript\footnote{coffescript.org
    coffescript ist eine Sprache die nach Javascript, genauer ECMAScript
    3, transcompiliert werden kann. Sie inspiriert sich von anderen
    Programmiersprachen wie Ruby oder Python und ist als syntactic sugar
    für JavaScript zu verstehen * Eine Ausgabe erfolgt von CasperJS}:
  geschrieben und nutzen die Funktionalität des \emph{tester} Modul von
  GhostJS zum abtasten der jeweiligen Seite und Funktionalität des
  \emph{utilis} Modul zum verarbeiten der Eingabeparameter und
  Ausgabe.\\Eine Aufnahme von Testabläufen ist mit den der
  Chrome-extension \emph{resurrectio}\footnote{https://github.com/ebrehault/resurrectio}
  möglich, diese gibt nach der Ausgabe den Quellcode eines Test als
  Javascript aus. \emph{resurrectio} wird seid 2013 nicht mehr gewartet
  sodass damit erstellte Tests nicht den vollen Funktionsumfang von
  Ghostjs \texttt{\$currentversion\ \ ,\ 1.0\ beta\ 3} ausnutzen können.
\item
  Test werden als JavaScript-Dateien gespeichert
\end{itemize}

Bildschrimfotos werden wie folgt erstellt.
https://github.com/casperjs/responsive-screenshots

\subsubsection{Umsetzung der Anforderungen Integration ins CI/CD
System}\label{umsetzung-der-anforderungen-integration-ins-cicd-system}

\begin{itemize}
\item
  Die Flusskontrolle \texttt{Die\ Logik\ ?} für Deployment- und
  Integrationsprozesse wird definiert mit Hilfe von ANT Skripten.
  Einzelne Aufgaben werden in sogenannte Pipelines verpackt.
  \texttt{Dieser\ Punkt\ kommt\ in\ den\ Flusstext,\ nicht\ Pflichtenheft}
\item
  Es wird in Go eine neue Pipelinegruppe angelegt mit je einer Pipeline
  pro zu testender Serverumgebung. Hier sind es 3 für das
  \emph{Integration}, \emph{Stage} und \emph{Echt}‌system.\\ Dazu werden
  neue Tags \texttt{XML-Knoten?} für Pipelines in die \texttt{go.xml}
  eingefügt. Der teaminternen Nomenklatur folgend heißen die Pipelines
  dann etwas ``UI.Test.ghostjs'' und ``UL.Test.ghostjs''
  \textgreater{}// Todo: Name klären
\item
  Es wird eine neue ghostjs.xml Datei erstellt die sämtliche ANT-Targets
  enthalten wird die für Regressionstests notwendig sind. Diese
  beinhalten u.a. :

  \begin{itemize}
  \itemsep1pt\parskip0pt\parsep0pt
  \item
    Auschecken aktueller Tests und Konfigurationsdateien aus der
    Versionsverwaltung ``svn.gravis.de/testing/trunk/ghostjs'' in das
    Basisverzeichnis der aktuellen Pipeline
  \item
    Vor-
  \item
    und nachbereitende Datenbankabfragen die Nutzerdaten von und für die
    Testnutzer der Regressionstests wie etwas Adressänderungen oder das
    löschen von Testkonten.
  \item
    Das Auslösen der Test
  \item
    Das Abholen und Aufbereitung der Testartefakte
  \item
    Die Auswertung der Testergebnisse
  \end{itemize}
\item
  Umgebungsvariablen werden in Go vergeben, die Basispfade für
  Testdateien und Artefakte festlegen, sowie Zugangsdaten für vor- und
  nachbereitende Datenbankzugriffe
\item
  Auf allen von Go verwalteten Servern muss ein ``Go-Agent'' ausgeführt
  werden. Dieser nimmt dann Arbeitsanweisungen von CI/CD System, aus den
  Pipelines entgegen und führt diese aus. Der Agent auf dem Server für
  Regressionstest muss in Go registriert werden.
\item
  Den Agenten muss eine oder mehrere Ressource(n) als Merkmal
  hinzugefügt werden

  \texttt{\textless{}agents\textgreater{}\ \ \ \ \ \textless{}agent\textgreater{}\ \ \ \ \ \ \ \ \ \textless{}name\textgreater{}mdsonline.stage.gravis.de.app2\textless{}/name\textgreater{}\ \ \ \ \ \ \ \ \ \textless{}ressources\textgreater{}stage\textless{}/ressources\textgreater{}\ \ \ \ \ \textless{}/agent\textgreater{}\ \textless{}agents\textgreater{}}
\item
  Zuweisung der ausführenden Ressource zu den jeweiligen Testing
  Pipelines.

  \texttt{\textless{}pipeline\textgreater{}\ \ \ \ \ \textless{}name\textgreater{}G.UI.testing.ghostjs\textless{}/name\textgreater{}\ \ \ \ \ \textless{}ressources\textgreater{}manager,stage\textless{}/ressources\textgreater{}\ \ \ \textless{}/pipeline\textgreater{}}
\item
  Das Auslösen der Test erfolgt über ein Unix-Shellkommando in einem
  ANT-Target. Hierbei wird casperJS mit Argumenten und Parametern
  aufgerufen. Hierbei können die Testtiefe als auch weitere Parameter
  übergebenwerden, etwa die gezielte Angebe von Suchtermini
  \texttt{\$ENV\_searchterms} oder Artikelnummern
  \texttt{\$ENV\_testporducts} die in die Test mit einbezogen werden
  sollen. Diese werden in Go als Umgebungsvariablen definiert und dann
  in das Unix-Shellkommando eingefügt.
  \texttt{casperjs\ Test\ pfad/zu/tests/*.js\ -\/-testdebth=\{\$ENV\_testdebth\}\ -\/-testproducts=\{\$ENV\_testporducts\}\ -\/-searchterms=\{\$ENV\_searchterms\}}.
  Damit ist bereits aus der Weboberfläche des CI/CD System eine hohe
  Anpassungsfähigkeit der Tests gewährleistet. Der Testleiter kann
  hierdurch ohne umständliche Code-Anpassungen die Testtiefe einstellen.
\item
  Es wird eine ``Pre-run'' Konfigurationsdatei für CasperJS die
  Zugangsdaten und wiederkehrende Prozeduren enthält sowie die
  Kommandozeilenparameter und -argumente aufbereitet und den folgenden
  Test in Variablen zur Verfügung stellt.
\item
  Ort für Artefakte bereiten
\item
  Ausgabe von Artefakten definieren (Screenshot)
\item
  Ausgabe von Artefakten definieren (XUnit-log)
\item
  Ausgabe von Artefakten definieren (Verbose Ausgabe von Casper
  \texttt{in\ Datei\ piepen?\ Kann\ ich\ ANSI-Farben\ behalten?})
\item
  \texttt{XUNIT-log\ schönen/aufbereiten?\ PHPUnit\ hat\ an\ dieser\ Stelle\ eine\ kleine\ Reportwebpage\ erstellt\ -\textgreater{}\ Will\ ich\ das,\ brauche\ ich\ das?}
\item
  Rückgabewerte an Go konkretisieren.
  \texttt{Schlägt\ die\ Pipeline\ fehl\ nur\ weil\ ein\ Test\ Fehlgeschlagen\ ist?}
\end{itemize}
